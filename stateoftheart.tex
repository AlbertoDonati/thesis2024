
\section{What are automation systems}
Automation systems are software tools that allow you to manage, monitor, and automate operations on a large number of servers. These systems can include features such as automatic provisioning, configuration management, service orchestration, performance monitoring, and troubleshooting.
An example of these systems is \textbf{Ansible}, which uses an agentless management model to connect to servers and perform automation tasks. Other examples include Puppet, Chef, and SaltStack, which offer similar features but use different approaches for server management.
These automation systems can be particularly useful in environments with a large number of servers, where manual management of each server would be inefficient and prone to errors. Through automation, organizations can improve operational efficiency, reduce errors, and free up IT staff to focus on other works.
Automation systems aims to simplify the execution of repetitive task.

\section{Who uses automation systems}

\section{When it makes sense (or doesn't make sense) to automate operations}

\section{What are the main systems}

\section{OpenSource and Closed Source, when OpenSource becomes Closed for some use cases}